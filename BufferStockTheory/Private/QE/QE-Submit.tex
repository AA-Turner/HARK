\documentclass[12pt]{letter}
\usepackage{latexsym}

\baselineskip=.15in
\pagestyle{empty}
\signature{Christopher D. Carroll \\ Department of Economics
\\ Johns Hopkins University \\ Baltimore, MD  21218-2685 
\\ 1-410-516-7602 (o) \\ 1-410-516-7600 (f) \\ ccarroll@jhu.edu}

\begin{document}

Editors,

As I read them, the Econometrica editor and referees had two main reasons for declining the paper:
\begin{enumerate}
\item Although the key results (the contraction mapping proof of convergence; existence of a target wealth ratio) have not been proven elsewhere, they are familiar from numerical work

\item The results are for a CRRA utility function, and perfectly general (as theorists like)
\end{enumerate}

I thought QE might be interested in publishing the paper because:
\begin{itemize}
\item For quantitative work (like that published in QE), virtually
  every paper uses CRRA utility (and the main features of
  CRRA that drive the results are the usual Inada conditions that are
  often imposed even without CRRA)
\item The paper provides a full analytical characterization of the parametric restrictions required for the results that the referees note are familiar from numerical work, and knowing when to expect these results to arise can be very useful for practitioners of the kinds of models often published in QE
\item The code archive that produces the paper's results (and is available on the author's website) could prove to be a useful starting point for authors interested in starting future research projects using this class of models
\end{itemize}

PS.  I have persuaded Adam Szeidl to submit his closely related paper proving the existence of an ergodic distribution of the wealth-to-permanent-income ratio in models of this class.  I would urge the editors to consider publishing the two papers together, as they are closely related.  

\closing{Sincerely,}

\ps{P.S.\  The paper as submitted is surely too long for publication in
  full; with advice about where to cut, it could be substantially shortened (with much material available in an online appendix).}


\end{document}


