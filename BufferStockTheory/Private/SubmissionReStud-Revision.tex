\documentclass[12pt]{letter}
\usepackage{latexsym}

\baselineskip=.15in
\pagestyle{empty}
\signature{Christopher D. Carroll \\ Department of Economics
\\ Johns Hopkins University \\ Baltimore, MD  21218-2685 
\\ 1-410-516-7602 (o) \\ 1-410-516-7600 (f) \\ ccarroll@jhu.edu}

\begin{document}
\begin{letter}{
Philipp Kircher, Editor \\
Review of Economic Studies
}

\opening{Dear Philipp,}

I believe Kjetil Storesletten handled my ``Theoretical Foundations''
paper when it was originally submitted to ReStud a few years ago (MS
10596); it was eventually accepted subject to revisions, but in the
course of revising I discovered several ways to make the results more
general, more accessible, and more powerful.  Those revisions took
much longer than intended (I was busy, among other things, working in
the Obama administration for a year), but the result is a paper that
is a lot stronger than the draft that was accepted.  However, Kjetil
tells me that when he handed his duties over to you, the two of you
decided that any papers that had not met the revision deadline (which
I have missed by a long margin!)  would be treated as new submissions.
The proper procedure, I think, is simply for me to submit the paper
again through the regular process, as I am doing?  (I believe I have
fully addressed the suggestions of the referees as well as the
editor's suggestions, but as this is not a resubmission I am not including a
formal reply to the referees' reports or the editor's letter).

In a related matter, I would urge you to reconsider a decision that
Kjetil made around time my paper was accepted (subject to revision);
Adam Szeidl had submitted a related paper that fits hand-in-glove with
mine and is quite short, proving the existence of an invariant
distribution in a model of precisely the kind I examine.  (In fact,
Adam's paper was written in response to an earlier draft of my paper
that speculated such a proof was possible but provided only simulation
evidence of invariance).  The two papers are so closely connected that
I believe each would benefit from association with the other; if you
were willing to publish Szeidl's paper, I would be able to cut a bit
of the discussion in the last section of my paper (presenting
simulation results) by referring more to Szeidl's paper.

\closing{Sincerely,}

\ps{The paper as submitted is presumably too long for publication in
  full; I anticipate that most of the appendix material should be
  available as an online supplement rather than published in its
  entirety.}

\end{letter}
\end{document}


